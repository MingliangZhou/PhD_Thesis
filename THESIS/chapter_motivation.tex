\section{Motivation for this thesis}

\subsection{Bulk properties}



\subsection{Motivations for this thesis}

QPG produced in the HI collisions is unstable and impossible to contain, and will decay into stable particles that can be observed by detectors. By measuring distribution and fluctuation of final stage particles, we could learn the initial stage conditions before the generation of QGP, as well as the hydrodynamical nature of the QGP expansion. The measurements conducted in this thesis can be divided into two directions: longitudinal and transverse. This section only gives an overview of motivations of this thesis, more details will be discussed in the corresponding sections.



\subsubsection{Longitudinal multiplicity measurements}

Before the collision of nucleus, the number of nucleon participants in the target and projectile fluctuates from event to event.  These longitudinal fluctuations directly seed the entropy production at very early time of the collision, well before the onset of the collective flow. Experimental measurements of these correlations provide a window into the space-time picture of the collective expansion as well as the medium properties that drives the expansion.

Longitudinal multiplicity correlation has been measured extensively in previous studies, however, many questions still remain to be addressed. Compared with previous measurements, this thesis highlights:
\begin{itemize}
\item Comprehensive measurement of two-particle multiplicity correlation between any two $\eta$ windows;
\item Suppression of backgrounds such as short-range correlation in a data-driven approach;
\item Comparison of the correlation strength from large to small collision systems;
\end{itemize}



\subsubsection{Transverse anisotropy measurements}

The non-uniform initial geometry of the collision zone also leads to an azimuthal anisotropy in the profile of the produced particles. The eccentricities in the initial geometry profile would eventually evolve to azimuthal anisotropy for the collision products. Due to event-by-event fluctuating positions of the participants, eccentricity also fluctuates from event to event. By measuring the flow probability distribution, we can probe the initial stage fluctuations.

Multi-particle cumulant method is designed to measure the flow fluctuation and also indicates the collectivity in the collision system. This thesis extends the cumulant measurements in the following aspects:
\begin{itemize}
\item Measurement carried out in different $\pT$ windows, attempting the disentangle the flow fluctuation from initial and final stages;
\item Comprehensive studies of correlated flow fluctuation between two different harmonics;
\item Introduce robust cumulant measurements to small systems, by effectively suppressing significant non-flow backgrounds;
\item Investigate the impact of centrality fluctuation effects in flow cumulant measurements;
\end{itemize}




\subsection{Outline}


