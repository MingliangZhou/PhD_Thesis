\section{Summary}
\label{chapter:summary}

This dissertation summarizes our recent endeavors to understand the initial stages and evolutions of heavy-ion collisions. Conclusions and outlooks of each measurement have been covered in previous sections. As a brief summary, highlights of this dissertation are the followings:
\begin{itemize}
\item \textbf{Transverse and longitudinal measurements} By utilizing similar decomposition techniques, we studied particle correlation in both the longitudinal and azimuthal directions, and gained many new insights towards a complete picture of fireball evolution. In the studies of MC models, we found that the asymmetry of participants from the initial stage is strongly correlated with the particle production of the final stage, which provided a good handle to measure the initial stage entropy density deposition in the longitudinal direction. By performing the measurements from large to small collision systems, a striking similarity of longitudinal fluctuation strength was observed in systems with dramatically different system sizes, which opened a new door to study the sub-nucleonic degree of freedom in the small collision system. Even though flow in the transverse plane has been studied extensively in recent years, we revisited some of the old measurements and proposed new explanations. For example, by defining different centralities, we were able to attribute the non-Gaussian fluctuations observed in the central collisions to the effects of centrality fluctuation, and helped achieving more robust flow measurements.
\item \textbf{Initial and final stages} Our measurements provided a handle to disentangle event-by-event fluctuation from the initial and final stages. By comparing the longitudinal multiplicity correlation between HIJING and AMPT MC models, we understood how the particle re-scattering effects in the final stage contributes to the multiplicity correlation. In the cumulant measurements, we repeated the analyses in four distinct $\pT$ ranges and proposed normalized cumulants to suppress the contributions from initial geometry. We were able to observe phenomenons which are directly related to the final stage dynamics. Furthermore, the comparison of the cumulant measurements from two colliding systems with slightly different nuclei size further constrained the estimation of shear viscosity over entropy density. All these results provided important experimental inputs to the hydrodynamical models.
\item \textbf{New observables and methods} We proposed new observables and methods that are more direct and sensitive to the physics mechanisms that we were interested in. Instead of using Pearson coefficient to study the multiplicity correlation, we designed a new observable which is robust to statistical fluctuation. A data-driven technique to suppress the short-range correlation was also proposed. Without such new observable and method, we were not able to discover similarities among the different collision systems. Furthermore, because of significantly larger backgrounds, measurements in extreme conditions need to be dealt cautiously. For example, subevent algorithm was invented to suppress the non-flow contributions in small systems, and flow fluctuations in ultra-central collisions were studied through a special designed method to evaluate the contributions from centrality fluctuations. All these new observables and methods have already witnessed their applications in other measurements.
\end{itemize}

But our journey does not end here. In Run 3 and Run 4, the ATLAS inner detector acceptance will be increased from $|\eta|=2.5$ to $|\eta|=4.0$, which provides a good opportunity to expand our longitudinal multiplicity measurements and help spot potential higher-order fluctuation modes. Furthermore, the luminosity increase in Run 3 and Run 4 will allow us to revisit the flow measurements in small systems, and high precision measurements of triangle flow $v_3$ and 6-particle cumulant will facilitate our understanding of the smallest droplet of QGP. Meanwhile, the undergoing geometry scan by the STAR experiments will provide the perfect conditions to further understand centrality fluctuation. The planned beam energy scan phase II might even help locate the critical end point in the QCD diagram. All these future developments, together with more precise and robust theoretical models, will no doubt push our understanding of QGP and QCD to a new level.

